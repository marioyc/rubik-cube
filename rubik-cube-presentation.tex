\documentclass{beamer}
\usepackage[utf8]{inputenc}
\usetheme{Warsaw}

\title{Finding Optimal Solutions to Rubik's cube}
\author{Mario Ynocente Castro}
\institute{Ecole polytechnique}

\begin{document}

\begin{frame}
\titlepage
\end{frame}

\begin{frame}
\frametitle{Cube class}

\begin{itemize}
\item
Representation of the cube as a 54 entries array.

\item
Functions:

\begin{itemize}
\item
Input: read(), initFromString(s)
\item
Output: toString(), print()
\item
Rotations: rotateLeft(times), rotateFront(times), rotateRight(times), rotateBack(times), rotateUp(times), rotateDown(times), rotate()
\item
Generating cubes: mixCube(moves), randomMove()
\item
Verification: check()
\end{itemize}

\end{itemize}

\end{frame}

\begin{frame}
\frametitle{IDASolverReturn class}

Used to return information about the solution

\begin{itemize}
\item
Number of moves in the optimal solution
\item
Number of calls to the search() function
\item
Time it took
\item
Sequence of moves in the solution
\end{itemize}
\end{frame}

\begin{frame}
\frametitle{IDASolverInterface class}

Provides the common parts to the three algorithms that use IDA*:

\begin{itemize}
\item
solve(Cube C): increasingly iterates over the possible depths until a solution is found
\item
search(int count, int maxMoves, Cube C):

\begin{itemize}
\item
Checks if a solution was found or the maximum depth was reached
\item
Cuts the search if the estimated number of moves is greater than the maximum number of moves
\item
Otherwise, it tries all possible next moves and calls itself
\end{itemize}
\end{itemize}
\end{frame}

\begin{frame}
\frametitle{IDASolverInterface class}

Two functions will be implemented by each algorithm:

\begin{itemize}
\item
getAlgorithm(): returns the algorithm's name
\item
estimate(Cube C): provides a lower bound for the remaining number of moves
\end{itemize}
\end{frame}

\begin{frame}
\frametitle{IDA1 class (Question 4)}

A simple lower bound is 0.
\end{frame}

\begin{frame}
\frametitle{IDA2 class (Questions 6,7)}

\begin{itemize}
\item
Calculates for each cube the minimum number of moves to place it at its correct position
\item
Each cube can be determined by the positions of two of its faces
\item
We do a BFS, and we case can save the results in a $54 \times 54 \times 54 \times 54$ matrix.
\end{itemize}
\end{frame}

\begin{frame}
\frametitle{IDA3 class (Questions 8,9)}

\begin{itemize}
\item
Finding Optimal Solutions to Rubik's Cube Using Pattern Databases" by Richard E. Korf

\item
One pattern datases that calculates the minimum number of moves to place all the corner cubes

\item
Two pattern databases that calculate the minimum number of moves to place a group of six edge cubes
\end{itemize}
\end{frame}

\begin{frame}
\frametitle{HeuristicsGenerator class}

\begin{itemize}
\item
3 Breadth First Searches
\item
We keep the states encoded as integers inside a queue

\begin{itemize}
\item
$88179840 = 8! \times 3^7$ corner states
\item
$42577920 = \frac{12!}{6!} \times 2^6$ edge states in each group
\end{itemize}
\end{itemize}
\end{frame}

\begin{frame}
\frametitle{CornersState class}

\begin{itemize}
\item
Keeps the positions of the three faces for 7 corners of the cube
\item
encode(): Turns the cube into a permutation of $0,1,\ldots,7$ and for each cube determine in which of its 3 possible orientations it is
\item
decode(): steps in the reverse order
\end{itemize}
\end{frame}

\begin{frame}
\frametitle{EdgesState class}

\begin{itemize}
\item
Keeps the positions of the two faces for the 6 edge cubes in a group
\item
encode(): Turns the cube into a permutation of 6 numbers from $0,1,\ldots,11$ and for each cube determine in which of its 2 possible orientations it is
\item
decode(): steps in the reverse order
\end{itemize}
\end{frame}

\begin{frame}
\frametitle{Some extra utilities}

\begin{itemize}
\item
testIDA(solver, moves)
\item
compareIDA(solver1, solver2, moves)
\item
compareIDA(solver1, solver2, s)
\item
measureAverageTime(solver, maxMoves, numIt)
\end{itemize}
\end{frame}

\begin{frame}
\frametitle{Average Time (IDA1)}

\begin{itemize}
\item
1 iteration(s), average time = 0.0187275553 milliseconds
\item
2 iteration(s), average time = 0.1304384461 milliseconds
\item
3 iteration(s), average time = 1.2987424571 milliseconds
\item
4 iteration(s), average time = 23.7366721729 milliseconds
\item
5 iteration(s), average time = 453.5737026717 milliseconds
\end{itemize}
\end{frame}

\begin{frame}
\frametitle{Average Time (IDA2)}

\begin{itemize}
\item
1 iteration(s), average time = 0.1548157628 milliseconds
\item
2 iteration(s), average time = 0.2683026750 milliseconds
\item
3 iteration(s), average time = 0.3011364937 milliseconds
\item
4 iteration(s), average time = 2.20671841264 milliseconds
\item
5 iteration(s), average time = 28.1426577772 milliseconds
\item
6 iteration(s), average time = 413.3850355594 milliseconds
\end{itemize}
\end{frame}

\begin{frame}
\frametitle{Average Time (IDA3)}

\begin{itemize}
\item
1 iteration(s), average time = 0.0588183479 milliseconds
\item
2 iteration(s), average time = 0.1293977463 milliseconds
\item
3 iteration(s), average time = 0.1742215336 milliseconds
\item
4 iteration(s), average time = 0.2337214848 milliseconds
\item
5 iteration(s), average time = 0.3156905944 milliseconds
\item
6 iteration(s), average time = 0.4754295848 milliseconds
\item
7 iteration(s), average time = 1.1059319282 milliseconds
\item
8 iteration(s), average time = 5.1622225588 milliseconds
\item
9 iteration(s), average time = 22.0955261797 milliseconds
\item
10 iteration(s), average time = 116.5331866548 milliseconds
\item
11 iteration(s), average time = 959.9939359772 milliseconds
\end{itemize}
\end{frame}

\end{document}